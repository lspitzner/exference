% Exference-LE.tex
\begin{hcarentry}{Exference}
\report{Lennart Spitzner}%11/16
\status{actively maintained}
\makeheader

Exference is a tool aimed at supporting developers writing Haskell code by
generating expressions from a type, e.g.\\
{\small
Input:
%
\begin{verbatim}
  (Show b) => (a -> b) -> [a] -> [String]
\end{verbatim}
%
Output:
%
\begin{verbatim}
  \ f1 -> fmap (show . f1)
\end{verbatim}
%
Input:
%
\begin{verbatim}
     (Monad m, Monad n)
  => ([a] -> b -> c) -> m [n a] -> m (n b)
  -> m (n c)
\end{verbatim}
%
Output:
%
\begin{verbatim}
  \ f1 -> liftA2 (\ hs i ->
    liftA2 (\ n os -> f1 os n) i (sequenceA hs))
\end{verbatim}
}

The algorithm does a proof search specialized to the Haskell type system.
In contrast to Djinn, the well known tool with the same general purpose,
Exference supports a larger subset of the Haskell type system - most
prominently type classes. The cost of this feature is that Exference makes
no promise regarding termination (because the problem becomes an undecidable
one; a draft of a proof can be found in the pdf below). Of course the
implementation applies a time-out.

There are two primary use-cases for Exference:

\begin{compactitem}
\item
  In combination with typed holes: The programmer can insert typed holes
  into the source code, retrieve the expected type from ghc and forward
  this type to Exference. If a solution, i.e.~an expression, is found
  and if it has the right semantics, it can be used to fill the typed
  hole.
\item
  As a type-class-aware search engine. For example, Exference is able to
  answer queries such as |Int -> Float|, where the common search engines like
  hoogle or hayoo are not of much use.
\end{compactitem}

The current implementation is functional and works well. The most
important aspect that still could use improvement is the performance, but it
would probably take a slightly improved approach for the core algorithm (and
thus a major rewrite of this project) to make significant gains.

The project is actively maintained; apart from occasional bug-fixing and
general maintenance/refactoring there are no major new features planned
currently.

Try it out by on IRC(freenode): exferenceBot is in \#haskell and \#exference.

\FurtherReading
{\small
\begin{compactitem}
  \item
    \url{https://github.com/lspitzner/exference}
  \item
    \url{https://github.com/lspitzner/exference/raw/master/exference.pdf}
\end{compactitem}
}
\end{hcarentry}
